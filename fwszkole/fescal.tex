\documentstyle[color,epsfig,psfig,12pt]{article}
%\textheight 23cm
%\textwidth 17cm
%\topmargin -1.cm
\sloppy
\definecolor{mycolor}{rgb}{1,0,0}
\begin{document}
%\thispagestyle{empty}
%\begin{center}
\noindent

Phys. Rev. Lett. {\bf 23}, 1415 (1969)
%Skalowanie Fenmana
\hline
\vspace{.5cm}

\begin{center}
\textcolor{mycolor}
{\Large \bf Very High-Energy Collisions of Hadrons}
\vspace{.5cm}

{\large Richard P. Feynman}%, Phys. Rev. Lett. {\bf 23}, 1415 (1969)}
 \vspace{.5cm}

%Streszczenie:
\begin{minipage}{12cm}
Zaproponowano przewidywania co do rozk\l ad\'{o}w p\c{e}d\'{o}w pod\l u\.{z}nych
w zderzeniach hadron\'{o}w skrajnie wysokich energii.
\end{minipage}
\end{center}
\vspace{1cm}


Prawdopodobnie $1/3$ ca\l kowitego przekroju czynnego przy wysokich energiach jest
elastyczna i 10\% daje si\c{e} interpretowa\'{c} jako dysocjacj\c{a}
dyfrakcyjn\c{a}. Reszta jest nieelastyczna. Zderzenia z udzia\l em kilku cz\c{a}stek
wychodz\c{a}cych badano szczeg\'{o}\l owo i z wyj\c{a}tkiem
cytowanych wy\.{z}ej przekroj\'{o}w elastycznych i dyfrakcyjnych wszystkie pozosta\l e
spadaj\c{a} z energi\c{a} (prawdopodobnie pot\c{e}gowo). Cz\c{e}\'{s}\'{c} sta\l a
przekroju nieelastycznego nie mo\.{z}e z nich pochodzi\'{c}.
A wiemy, \.{z}e dla tych energii wi\c{e}kszo\'{s}\'{c} zderze\'{n} prowadzi
do powstawania du\.{z}ej liczby cz\c{a}stek wt\'{o}rnych
(najpewniej ich krotno\'{s}\'{c} ro\'{s}nie z energi\c{a} logarytmicznie).
Zderzenia takie nie by\l y badane szczeg\'{o}lnie intensywnie poniewa\.{z}
z wielk\c{a} ilo\'{s}ci\c{a} produkowaneych cz\c{a}stek
tak wiele zmiennych i ich kombinacji nale\.{z}a\l oby zbada\'{c}, \.{z}e
nikt nie nie wie jak dane te uj\c{a}, zanalizowa\'{c} i w ko\'{n}cu zaprezentowa\'{c}
rezultaty.

Celem tej pracy jest poczynienie sugestii co do zachowania sie
ca\l kowitego nieelastycznego przekroju czynnego
i mo\.{z}liwo\'{s}ci wyciagania o tym wniosk\'{o}w
z danych zmierzonych przy r\'{o}\.{z}nych energiach.
Sugestie te wzi\c{e}\l y si\c{e} z rozwa\.{z}a\'{n} teoretycznych
na r\'{o}\.{z}ne tematy i nie reprezentuj\c{a} jakiego\'{s} jednego, konkretnego modelu.
S\c{a} one rezultatem po\l \c{a}czenia w\l asno\'{s}ci
 relatywistycznych, kwantowomechanicznych z pewnymi obserwacjami
 empirycznymi (a s\c{a} nimi realne istnienie
 w jakim\'{s} tam stopniu biegun\'{o}w Reggego, a co za tym idzie s\l uszno\'{s}\'{c}
 jego teorii rozprosze\'{n}, sta\l o\'{s}\'{c} ca\l kowitego przekroju czynnego
 i niezale\.{z}no\'{s}\'{c} od energii oddzia\l ywania rozk\l ad\'{o}w p\c{e}du
 poprzecznego) i wydaj\c{a} si\c{e} by\'{c} niezale\.{z}ne od modelu oddzia\l ywania.

Pisanie tej notatki (tego artyku\l u) nie jest dla mnie rzecz\c{a} prost\c{a}, poniewa\.{z}
ze swej natury nie jest to praca o charakterze dedukcyjnym, a indukcyjnym raczej.
Jestem bardziej pewny swoich wniosk\'{o}w ko\'{n}cowych, ni\.{z} ka\.{z}dego z
argument\'{o}w do nich wiod\c{a}cych, a to g\l \'{o}wnie dla ich (tych wniosk\'{o}w)
wewn\c{e}trznej sp\'{o}jno\'{s}ci, kt\'{o}ra zdumiewa mnie i przewy\.{z}sza
sp\'{o}jno\'{s}\'{c} argument\'{o}w za nimi przemawiaj\c{a}cych.

W pracy tej jedynie zarysowane b\c{e}da\c{a} podtawowe za\l o\.{z}enia mojego
wnioskowania. By\'{c} mo\.{z}e w przysz\l ej publikacji b\c{e}d\c{e} m\'{o}g{\l }
poda\'{c} wi\c{e}cej szczeg\'{o}\l \'{o}w. (Szerszy i bardziej szczeg\'{o}\l owy
opis znajdzie si\c{e} w pracy: R.P. Feynman w {\it Proceedings of the Third
Topical Conference on High Energy Collisions of Hadrons},
Stony Brook, N.Y.)

Zak\l adaj\c{a}c, \.{z}e p\c{e}dy poprzeczne s\c{a} ograniczone w spos\'{o}b
niezale\.{z}ny od energii zderzaj\c{a}cych si\c{e} cz\c{a}stek ($s=2W^2$)
z teorii pole wynika istnienie w granicy bardzo du\.{z}ych $W$ zmiennych w\l a\'{s}ciwych
dla opisu produkowanych cz\c{a}stek przy por\'{o}wnywaniu w uk\l adzie
\'{s}rodka masy
eksperymen\'{o}w z r\'{o}\.{z}n\c{a} energi\c{a} $W$.
Zmiennymi tymi s\c{a} stosunek pod\l u\.{z}nego p\c{e}du cz\c{a}stki $P_z$ do maksymalnego
mo\.{z}liwego do osi\c{a}gni\c{e}cia $W$
$$
x~=~P_z/W
$$
i p\c{d} poprzeczny $Q$ (mierzony w jednostkach p\c{e}du).
R\'{o}\.{z}niczkowe przekroje czynne wyra\.{z}one w takich zmiennych b\c{e}d\c{a}
mia\l y proste w\l asno\'{s}ci (jako funkcje $W$).

Po pierwsze trzeba rozr\'{o}\.{z}ni\'{c} reakcje {\bf ekskluzywne} od {\bf
inkluzywnych}. W eksperymentach ekskluzywnych pytamy o produkcje konkretnych cz\c{a}stek
o okre\'{s}lonych $x$ i $Q$ i \underline {\.{z}adnych} innych. Przyk\l adem jest
dwucia\l owa reacja wymiany \l adunku. Typow\c{a} reakcj\c{e} ekskluzywn\c{a}
mo\.{z}na zapisa\'{c} jako
$$
A~+~B~\rightarrow~\sum \limits_{1}^{n} C_i~+~\sum \limits_{1}^{n'} D_i~~~,
$$
gdzi $A$ leci na prawo, $B$ na lewo, a $C_1, C_2,\ldots,C_n$ s\c{a} okre\'{s}lonymi
cz\c{a}stkami z okre\'{s}lonymi $Q$ i $x$ poruszaj\c{a}cymi si\c{e} na prawo ($x>0$),
podczas gdy $D_1, D_2,\ldots,D_n$ poruszaj\c{a} si\c{e} na lewo ($x<0$).
W takim przypadku przekr\'{o}j czynny powinien zmienia\'{c} si\c{e} przy dostatecznie
wysokich $W$ jak
\mbox{$
s^{2 \:\alpha(t)\:-\:2}
$}
czyli jak $
\left(W^2\right)^{2 \:\alpha(t)\:-\:2}
$, gdzie $\alpha(t)$ jest nachyleniem $\alpha$ najwy\.{w}szej trajektorii Reggego
mog\c{a}cej przekszta\l ci\'{c} liczby kwantowe cz\c{a}stki $A$ w liczby kwantowe ca\l ego
zespo\l u cz\c{a}stek $C$, a $t$ jest r\'{o}\.{z}nic\c{a} p\c{e}d\'{o}w cz\c{a}stki $A$
i sumy $C$.

Jest to oczywistym wnioskiem z teorii Reggego i powinno obowi\c{a}zywa\'{c} w przybli\.{z}eniu
tak jak obowi\c{a}zuje dla reakcji dwucia\l owych. Dodatkowo w tym miejscu trzeba wyja\'{s}ni\'{c}
u\.{z}ycie zmiennej $x$ dla por\'{o}wnania wynik\'{o}w z eksperyment\'{o}w przy r\'{o}\.{z}nych
energiach. Je\'{s}li \.{z}adne liczby kwantowe nie musz\c{a} by\'{c} w oddzia\l ywaniu
wymieniane, ca\l a grupa cz\c{a}stek $C$ ma takie same liczby kwantowe jak cz\c{a}stka $A$,
a wtedy $C$ mo\.{z}e powst\'{c} z $A$ na drodze dysocjacji dyfrakcyjnej i odpowiedni
przekr\'{o}j czynny powinien osi\c{a}ga\'{c} w granicy sta\l y stosunek do
przekroju elastycznego dla tego samego $t$ (co oznacza, \.{z}e powinien pozostawa\'{c} sta\l ym,
gdy sta\l ym pozostaje przekr\'{o}j elastyczny).

Eksperyment inkluzywny to taki, w kty\'{o}rym interesuje nas specjalny typ cz\c{a}stki,
jej $x$ i $Q$ w stanie ko\'{n}cowym, i nie obchodzi nas, co jeszcze wytworzone zostanie w
konkretnym oddzia\l ywaniu. Przyk\l adem mo\.{z}e by\'{c} \'{s}rednia liczba $K^+$
wyprodukowanych z okre\'{s}lonymi $x$ i $Q$ w reakcji $pp$. Taki przekr\'{o}j czynny powinien
by\'{c} sta\l y, gdy $W \rightarrow \infty$.

Jak mo\.{z}na to ze sob\c{a} pogodzi\'{c}?  Dlaczego przekr\'{o}j czynny maleje dla przyk\l adu
w reakcjach dwucia\l owych, w kt\'{o}rych dochodzi do wymiany
trzeciej sk\l adowej izospinu? Poniewa\.{z} w takich warunkach pr\c{a}d trzeciej
skladowej izospinu musi nagle zmieni\c{c} kierunek (z lewa na prawo). I z tego powodu je\'{s}li
jakiekolwiek pole skojarzone jest z tym \l adunkiem spodziewa\'{c} si\c{e} nale\.{z}y
emisji promieniowania (w spos\'{o}b analogiczny do promieniowania hamowania).
Aby eksperyment by{\l } ekskluzywny (powiedzmy czysty proces dwucia\l owy),
wymagamy aby promieniowanie takie nie pojawi\l o si\c{e}, warunek coraz bardziej trudny
do spe\l nienia, gdy energia wzrasta i zmiana kierunku pr\c{a}du bardzoej gwa\l towna.

Prowadzi to do przypuszczenia, \.{z}e w zderzeniach produkuje si\c{e}  du\.{z}o
cz\c{a}stek
w szerokim przedziale $x$, lecz ich charakterystyki dla ma\l ych $x$ s\c{a}
\l atwe do zobrazowania (envision).
Na skutek transformacji Lorentza emitowane pole staje si\c{e} ze wzrostem energii $W$
 coraz w\c{e}\.{z}sze (w kierunku $z$).Energia tego pola jest zatem funkcj\c{a} $\delta$
w $z$. Po zanalizowaniu tego fourierowsko uzyskuje si\c{e}, \.{z}e energi pola jest
roz\l o\.{z}ona r\'{o}wnomiernie w p\c{e}dzie wzd\l u\.{z} kierunku $z$: $dP_z$.
Poniewa\.{z} cz\c{a}stka o masie $\mu$ niesie energi\c{e} $E=(\mu^2+P_z^2+Q^2)^{1/2}$,
je\'{s}li przyj\c{a}\'{c}, \.{z}e energia rozk\l ada si\c{e} na r\'{o}\.{z}ne rodzaje
cz\c{a}stek w sta\l ym stosunku (niezale\.{z}nym od energii $W$),
okaz\.{z}e si\c{e}, \.{z}e \'{s}rednio cz\c{a}steki
ka\.{z}dego rodzaju o okre\'{s}lonym $Q$ roz\l o\.{z}one s\c{a} dla niezbyt du\.{z}ych $x$
jak $dP_z/E$.
Oznacza to, \.{z}e prawdopodobie\'{n}stwo znalezienia w\'{s}r\'{o}d nowowytworzonych cz\c{a}stek
cz\c{a}stki typu $i$ o p\c{e}dzie poprzecznym $Q$ i masie $\mu_i$ ma posta\'{c}
%$$
%{f_i\left(Q,\:P_z/W\right)\:dP_z\:d^2Q \over \left( \mu_i^2\:+\:Q^2\:+\:P_z^2 \right)^{1/2}}~~~,
%$$
\mbox{$
{f_i\left(Q,\:P_z/W\right)\:dP_z\:d^2Q / \left( \mu_i^2\:+\:Q^2\:+\:P_z^2 \right)^{1/2}}$},
gdzie $f_i(Q,\:x)$ jest ostatecznie niezale\.{z}ne od $W$ i ma granic\c{e}
$F_i(Q)$ dla ma\l ych $x$. Oczywi\'{s}cie,
gdy $W \rightarrow \infty$ dla ka\.{z}dego sko\'{n}czonego $x$
$dP_z/E$ przechodzi w $dx/x$.

Z powodu tego ($dx/x$) zachowania \'{s}rednia ca\l kowita liczba cz\c{a}stek
(krotno\'{s}\'{c}) ka\.{z}dego rodzaju cz\c{a}stek ro\'{s}nie z $W$ logarytmicznie.
I nie musimy wcale zastanawia\'{c} si\c{e} nad tym co tak naprawde emituje te cz\c{a}stki
i sk\c{a}d one pochodz\c{a}, poniewa\.{z} podane w\l a\'{s}nie wnioski wcale od tego
nie zale\.{z}\c{a}.
Je\'{s}li wyobrazi\'{c} sobie istnienie jakich\'{s} niezale\.{z}nych od siebie
pierwotnie wyemitowanych
obiekt\'{o}w, kt\'{o}re rozpadaj\c{a}c si\c{e} produkuj\c{a} interesuj\c{a}ce nas
cz\c{a}stki ich srednia liczba $\bar n$ tak\.{z} b\c{e}dzie ros\l a logarytmicznie
z energi\c{a} i szansa, \.{z}e nie b\c{e}dzie wyemitowane nic jest (zgodnie z rozk\l adem
Poissona) r\'{o}wna e$^{- \bar n}$ co w rezultacie zanika\'{c} b\c{e}dzie
z energi\c{a} pot\c{e}gowo prowadz\c{a}c do wzor\'{o}w teorii Reggego, kt\'{o}re
przypuszczamy, \.{z}e obowi\c{a}zuj\c{a} dla takich reakcji ekskluzywnych.

Mo\.{z}emy rozci\c{a}gn\c{a} t\c{e} ide\c{e} na inne amplitudy, w kt\'{o}rych pojawia si\c{e}
podobne $\bar n$. W szczeg\'{o}lno\'{s}ci prawdobodobie\'{n}stwo reakcji
\mbox{$A~+B~\rightarrow ~C~{\rm cokolwiek}$} powinno zmienia\'{c} si\c{e} jak
\mbox{$
    \left(   1~-~x_C \right)^{1-2\alpha(t)}\:dx_C
    $},
gdzie $C$ porusza si\c{e} w prawo unosz\c{a}c prawie ca\l y p\c{e}d $A$ (co oznacza, \.{z}e
$(1-x_C)$ jest ma\l e). $\alpha(t)$ oznacza tutaj najwy\.{z}sz\c{a} trajektorie
(\underline{z wy\l \c{a}czeniem} pomeronu chukon)
mog\c{a}c\c{a} przenie\'{s}\'{c} liczby kwantowe
(i kwadrat przekazu p\c{e}du) niezb\c{e}dny by zmieni\'{c} $A$ w $C$.

Tak wi\c{e}c funkcja opisuj\c{a}ca trajektori\c{e} Reggego $\alpha(t)$ pojawia si\c{e} nie tylko
przy oddzia\l ywaniach (jako $s^{2\alpha -2}$), ale tak\.{z}e i przy emisji nowych cz\c{a}stek
przypominaj\c{a}c \'{s}cis\l y zwi\c{a}zek odzia\l ywa\'{n} wirtualnych i emisji realnych
cz\c astek (w nawi\c{a}zaniu do pogl\c{a}d\'{o}w Yukawy).

Na koniec jedno zdanie o tych szczeg\'{o}lnych reakcjach, ekskluzywnych jedynie cz\c{e}\'{s}ciowo,
w kt\'{o}rych hadron musi zmieni\'{c} kierunek zprawego na lewy
(przenosz\c{a}c jednocze\'{s}nie sw\'{o}j fermionowy po\l\'{o}wkowospinowy charakter).
Przekr\'{o}j czynny w takim przypadku powinien zachowywa\'{c} si\c{e} jak $1/s$.
Co do s\l uszno\'{s}ci tej ostatniej konkluzji, to jestem jej
mniej pewny, ni\.{z} w odniesieniu do pozosta\l ych.












\end{document}
